%-------------------------------------------------------------------------------
%	SECTION TITLE
%-------------------------------------------------------------------------------
\cvsection{Work Experience}


%-------------------------------------------------------------------------------
%	CONTENT
%-------------------------------------------------------------------------------
\begin{cventries}

%---------------------------------------------------------
  \cventry
    {Site Reliability Engineer Intern} % Job title
    {Google} % Organization
    {Dublin (Remote), Ireland} % Location
    {Jun. 2022 - Sep. 2022} % Date(s)
    {
      \begin{cvitems} % Description(s) of tasks/responsibilities
        \item {Enhanced the infrastructure by providing safe and reliable progressive rollouts for the critical services.}
        \item {Visualised the rollout for GCP services to illustrate the structure and analyse the risks.}
        \item {Designed and implemented an visualiser with Python that takes in rollout information and outputs an SVG visualisation.}
      \end{cvitems}
    }

%---------------------------------------------------------
  \cventry
    {Site Reliability Engineer Intern} % Job title
    {Google} % Organization
    {London (Remote), UK} % Location
    {Jun. 2021 - Sep. 2021} % Date(s)
    {
      \begin{cvitems} % Description(s) of tasks/responsibilities
        \item {Developed infrastructure to reuse dozens of checks about Google’s production services (previously run only in a separate environment, with daily notifications), so that they are now integrated with Tricorder, the program analysis platform used for code reviews at Google.}
        \item {Significantly enhanced development efficiency by accelerating the feedback loop and detecting production failures prior to submitting changes.}
        \item {Worked on this project entailed using the Golang, non-trivial use of Bazel/Starlark custom rules and macros, ProtoBufs, and several other internal Google systems.}
        \item {For definitions of Tricorder and other internal Google tooling: \href{https://opensource.google/docs/glossary/}{https://opensource.google/docs/glossary/}.}
        % \item {Developed, tested, and deployed infrastructure to enable the continuous validation platform, which were able to run on an hourly basis, running the validation at presubmit stage.}
        % \item {Implemented a validator for the continuous validation platform that can detect mistakenly-placed static contents in the package and found 35 failed validation out of 2100 services.}
        % \item {Wrote the whole implementation doc and unblocked many problems by discussing with other colleagues and contacting the owner team.}
      \end{cvitems}
    }

%---------------------------------------------------------
  \cventry
    {Full Stack Intern} % Job title
    {Inductosense Ltd.} % Organization
    {Bristol, UK} % Location
    {Jun. 2020 - Sep. 2020} % Date(s)
    {
      \begin{cvitems} % Description(s) of tasks/responsibilities
        \item {Constructed pages and components to detect devices and edit schedule configuration for the Electron app with React and TypeScript.}
        \item {Implemented device and schedule-related API endpoints and relevant tests used by the Electron app with C\# and .NET.}
        \item {Developed data-format transform utility to solve the issue with incompatible frontend and backend design.}
      \end{cvitems}
    }

%---------------------------------------------------------
%   \cventry
%     {Engineering Intern} % Job title
%     {Goldman Sachs} % Organization
%     {London (Remote), UK} % Location
%     {May. 2020 - May. 2020} % Date(s)
%     {
%       \begin{cvitems} % Description(s) of tasks/responsibilities
%         \item {Learnt and presented how Goldman Sachs utilises Cloud/AWS to improve and expand its business with digital platforms, e.g., Marcus, Marquee, ClearFactr, etc.}
%         \item {Learnt how the frontend team of Goldman Sachs builds user-friendly products to simplify client experiences and generate new revenue streams.}
%         \item {Mocked technical interview with Java and Python.}
%       \end{cvitems}
%     }


%---------------------------------------------------------
\end{cventries}
